Motivated by the popularization of virtual reality technology, \citeonline{siu2020virtual} developed a white cane to be used by BVI users in a virtual environment. Their purpose was to make virtual reality applications available for BVI users. 

The traditional white cane transmits three sources of information to the user: detection of obstacles, surface topography and foot placement preview. In their work, these sources of information were transmitted through sounds or haptics \cite{siu2020virtual}, which would be defined based on the cane position in the virtual environment. For obstacle detection, the cane was built with a three-degree-of-freedom brake mechanism that would stop the movement when the cane hit an obstacle. A coil actuator was used to simulate surface properties. Lastly, a wave-based acoustic simulation was used to render geometry-aware sound effects in other to give the user a sense of the surroundings (echo localization).

In order to evaluate their proposal, the authors performed an experiment where the participants had to play a “scavenger hunt” using an HTC Vive system. During the experiment, each participant had two tasks: collect targets along the way (primary task) and avoid virtual obstacles and walls (secondary task). The targets appeared, one at a time, once the previous target was collected, and they emitted a sound that acted like an audio beacon for the participant. The obstacles did not emit any sound as a beacon, but the participant could detect it by the shape and the noise it emitted when in contact with the cane. The experiment was performed with 8 blind users (4 female, 4 male) from 25 to 70 years old. All of them did a training section where the virtual environment was presented. 

Among the relevant findings of \citeonline{siu2020virtual} is that not all the participants reacted the same to a particular stimulus. The vibration of the cane was considered confusing by some participants, while others were familiar with it. This difference affected the performance of the participants. The ones that had already used vibrating devices performed better. It shows that user's previous experiences can impact their performance in the virtual environment.

Another interesting observation was that, similar to what happens in the real world, it was easier for the participants to navigate in larger areas than in tight spaces. Moreover, the authors observed that the participants focused their attention on the primary task, without freely exploring the environment, which might have impacted the low time to achieve the goal and the low number of obstacle hits. 

Among the limitations pointed out by the authors is the lack of feedback possibilities for situations such as when the obstacle contacts a point along with the cane, not the tip of it, and the fact that the brake system did not stop the participant when he/she walked forwards toward a wall.

Comparing the work of \citeonline{siu2020virtual} to this work, \citeonline{siu2020virtual} were focused on providing mechanisms for a BVI user to navigate inside virtual environments. In this work, the purpose is to use the virtual environment to collect data about how the BVI user would navigate in a real environment. Another difference is in the functioning of the virtual cane, which in this work is limited to vibration, with no brake system, as the BVI user does not need to touch the environment with it. One common observation of both works is the sound importance for the BVI guidance and the need to use high-quality spatialized audio to increase the realism of the virtual environment.