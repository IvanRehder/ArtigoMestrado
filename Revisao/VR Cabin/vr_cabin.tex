% \lipsum[2-4]

The use of virtual reality for design purposes is not new. It has been studied and evaluated in several areas, including Aeronautics. \citeonline{moerland2021application} investigated using virtual reality during the aircraft cabin’s design process with the purpose of facilitating the communication between the design team and the client. 

The cabin design process (Figure \ref{fig:simplified_cabin_process}) is often said to be complex because it involves several stakeholders, each with his/her own set of preferences and requirements. According to the authors, the time needed to satisfy the multiple demands tends to be long, and the process usually requires building multiple mock-ups and attending many meetings with the stakeholders. 

\begin{figure}[!htbp]
    \centering    
    \tikzstyle{hexag} = [regular polygon, regular polygon sides=6, minimum size = 3cm, inner sep = 0cm, xshift = 1.5cm]
    \tikzstyle{hexagText} = [text = white, xshift = 1.5cm]
    \tikzstyle{legenda} = [fill = white, line width = 0.25mm]
    \tikzstyle{--} = [line width = 0.25mm]
    
    \tikzstyle{arrow} = [rounded corners, line width = 1mm, -to]
    \tikzstyle{arrow_blue} = [cor5, arrow]
    
    \resizebox{\linewidth}{!}{
    \begin{tikzpicture}[node distance=1.75cm]
        \centering    

        \node [hexag, draw = cor1, fill = cor1] (emphatize) {};
        \node [hexagText] (emphatizeText) {};
        
        \node [hexag, draw = cor2, fill = cor2, right of = emphatize] (define) {};
        \node [hexagText, right of = emphatize] (defineText) {DEFINE};
        
        \node(inicioFlecha1) [xshift = 3.75cm, yshift = 2.25cm] {};
        \node(fimFlecha1) [right of = emphatize, xshift = 1cm, yshift = 1.2cm] {};
        \node(inicioFlecha2) [xshift = 2.75cm, yshift = 1.5cm] {};
        \node(fimFlecha2) [right of = emphatize, xshift = 0.5cm, yshift = 0.8cm] {};
        \node(inicioFlecha3) [xshift = 1.7cm, yshift = 0.8cm] {};
        \node(fimFlecha3) [right of = emphatize, xshift = 0.25cm, yshift = 0.4cm] {};
        \node(inicioFlecha4) [xshift = 2cm, yshift = -0.5cm] {};
        \node(fimFlecha4) [right of = emphatize, xshift = 0.15cm, yshift = 0cm] {};
        \node(inicioFlecha5) [xshift = 2.77cm, yshift = -1.6cm] {};
        \node(fimFlecha5) [right of = emphatize, xshift = 0.25cm, yshift = -0.5cm] {};
        \node(inicioFlecha6) [xshift = 3.4cm, yshift = -2.2cm] {};
        \node(fimFlecha6) [right of = emphatize, xshift = 0.8cm, yshift = -1.2cm] {};
        
        \draw[arrow] (inicioFlecha1) .. controls ++(0.3,-0.5) .. (fimFlecha1);
        \draw[arrow] (inicioFlecha2) .. controls ++(0.5,-0.15) .. (fimFlecha2);
        \draw[arrow] (inicioFlecha3) .. controls ++(0.75,0) .. (fimFlecha3);
        \draw[arrow] (inicioFlecha4) .. controls ++(0.5,0.25) .. (fimFlecha4);
        \draw[arrow] (inicioFlecha5) .. controls ++(0.25,0.5) .. (fimFlecha5);
        \draw[arrow] (inicioFlecha6) .. controls ++(0.25,0.5) .. (fimFlecha6);
        
        \node [hexag, draw = cor3, fill = cor3, right of = define] (ideate) {};
        \node [hexagText, right of = define] (ideateText) {IDEATE};
        
        \node(inicioDFlecha1) [above of = define, xshift = 0.25cm, yshift = -0.1cm] {};
        \node(fimDFlecha1) [above of = ideate, xshift = -0.25cm, yshift = -0.1cm] {};
        \node(inicioIFlecha1) [below of = ideate, xshift = -0.25cm, yshift = 0.1cm] {};
        \node(fimIFlecha1) [below of = define, xshift = 0.25cm, yshift = 0.1cm] {};
        
        \draw[arrow, draw = cor3] (inicioDFlecha1.south) .. controls ++(0.25,0.5) and ++(-0.75,0.5).. (fimDFlecha1.south);
        \draw[arrow, draw = cor3] (inicioIFlecha1.north) .. controls ++(-0.25,-0.5) and ++(0.75,-0.5).. (fimIFlecha1.north);
        
        \node [hexag, draw = cor4, fill = cor4, right of = ideate] (prototype) {};
        \node [hexagText, right of = ideate] (prototypeText) {PROTOTYPE};
        
        \node(inicioIFlecha2) [above of = ideate, xshift = 0.25cm, yshift = -0.2cm] {};
        \node(fimIFlecha2) [above of = prototype, xshift = -0.25cm, yshift = -0.2cm] {};
        \node(inicioPFlecha1) [below of = prototype, xshift = -0.25cm, yshift = 0.2cm] {};
        \node(fimPFlecha1) [below of = ideate, xshift = 0.25cm, yshift = 0.2cm] {};
        \node(inicioPFlecha2) [above of = prototype, xshift = 0.25cm, yshift = -0.2cm] {};
        \node(fimPFlecha2) [above of = define, xshift = -0.25cm, yshift = -0.2cm] {};
        
        \draw[arrow, draw = cor4] (inicioIFlecha2.center) .. controls ++(0.25,0.5) and ++(-0.75,0.5).. (fimIFlecha2.east);
        \draw[arrow, draw = cor4] (inicioPFlecha1.east) .. controls ++(-0.25,-0.5) and ++(0.75,-0.5).. (fimPFlecha1.center);
        \draw[arrow, draw = cor1] (inicioPFlecha2.west) .. controls ++(-1,1.75) and ++(1.2,1.5).. (fimPFlecha2);
        
        
        \node [hexag, draw = cor5, fill = cor5, right of = prototype] (assess) {};
        
        \node(inicioPFlecha3) [above of = prototype, xshift = 0.25cm, yshift = -0.2cm] {};
        \node(fimPFlecha3) [above of = assess, xshift = -0.25cm, yshift = -0.2cm] {};
        \node(inicioAFlecha1) [below of = assess, xshift = -0.25cm, yshift = 0.2cm] {};
        \node(fimAFlecha1) [below of = prototype, xshift = 0.25cm, yshift = 0.2cm] {};
        \node(inicioAFlecha2) [below of = assess, xshift = 0cm, yshift = 0.2cm] {};
        \node(fimAFlecha2) [below of = ideate, xshift = -0.25cm, yshift = 0.1cm] {};
        \node(inicioAFlecha3) [below of = assess, xshift = 0.25cm, yshift = 0.2cm] {};
        \node(fimAFlecha3) [below of = define, xshift = -0.25cm, yshift = 0.1cm] {};
        
        \draw[arrow, draw = cor5] (inicioPFlecha3.center) .. controls ++(0.5,1) and ++(-0.75,1).. (fimPFlecha3.center);
        \draw[arrow, draw = cor5] (inicioAFlecha1.center) .. controls ++(-0.25,-0.5) and ++(0.75,-1).. (fimAFlecha1.west);
        \draw[arrow, draw = cor2] (inicioAFlecha2.center) .. controls ++(-0.25,-1.25) and ++(0.75,-1).. (fimAFlecha2.south east);
        \draw[arrow, draw = cor1] (inicioAFlecha3.center) .. controls ++(-0.25,-1.75) and ++(1.2,-1.5).. (fimAFlecha3.north);
        
        \node [hexagText, right of = prototype] (assessText) {ASSESS};
        

        
        
    \end{tikzpicture}
    }
    \caption{Simplified cabin design process (Adapted from \citeonline{moerland2021application}).}
    \label{fig:simplified_cabin_process}
\end{figure}

\citeonline{moerland2021application} proposed to anticipate the involvement of the final users based on co-design. In their proposal, the users can influence the product's development from the beginning, as shown in Figure \ref{fig:user_involvement}. However, for the involvement to happen, a communication channel needed to be established, and it was done using virtual reality.

\begin{figure}[!htbp]
    \centering    
    \tikzstyle{hexag} = [regular polygon, regular polygon sides=6, minimum size = 3cm, inner sep = 0cm, xshift = 1.5cm]
    \tikzstyle{hexagText} = [text = white, xshift = 1.5cm]
    \tikzstyle{legenda} = [fill = white, line width = 0.25mm]
    \tikzstyle{--} = [line width = 0.25mm]
    
    \resizebox{\linewidth}{!}{
    \begin{tikzpicture}[node distance=1.75cm]
        \centering    

        \node [hexag, draw = cor1, fill = cor1] {};
        \node [hexagText] (emphatize) {EMPHATIZE};
        %\node [regular polygon, regular polygon sides=6, text = white, draw = cor1, minimum size = 5cm, fill = cor1] (emphatize) {EMPHATIZE};
        \node [hexag, draw = cor2, fill = cor2, right of = emphatize] (define) {};
        \node [hexagText, right of = emphatize] (defineText) {DEFINE};
        %\node [regular polygon, regular polygon sides=6, text = white, draw = cor2, minimum size = 5cm, fill = cor2, right of = emphatize] (define) {DEFINE};
        \node [hexag, draw = cor3, fill = cor3, right of = define] (ideate) {};
        \node [hexagText, right of = define] (ideateText) {IDEATE};
        %\node [regular polygon, regular polygon sides=6, text = white, draw = cor3, minimum size = 5cm, fill = cor3, right of = define] (ideate) {IDEATE};
        \node [hexag, draw = cor4, fill = cor4, right of = ideate] (prototype) {};
        \node [hexagText, right of = ideate] (prototypeText) {PROTOTYPE};
        %\node [regular polygon, regular polygon sides=6, text = white, draw = cor4, minimum size = 5cm, fill = cor4, right of = ideate] (prototype) {PROTOTYPE};
        \node [hexag, draw = cor5, fill = cor5, right of = prototype] (assess) {};
        \node [hexagText, right of = prototype] (assessText) {ASSESS};
        %\node [regular polygon, regular polygon sides=6, text = white, draw = cor5, minimum size = 5cm, fill = cor5, right of = prototype] (assess) {ASSESS};

        \node(insertNorth1) [right of = emphatize, xshift = -0.125cm, yshift = 0.25cm] {};
        \node(insertNorthWest1) [above of = insertNorth1, left of = insertNorth1, xshift = 0.75cm] {};
        \node(insertNorthEast1) [above of = insertNorth1, right of = insertNorth1, xshift = -0.75cm] {};
        \draw[--] (insertNorthWest1.center) to (insertNorth1.center) to (insertNorthEast1.center);
        
        \node(insertSouth1) [right of = emphatize, xshift = -0.125cm, yshift = -0.25cm] {};
        \node(insertSouthWest1) [below of = insertSouth1, left of = insertSouth1, xshift = 0.75cm] {};
        \node(insertSouthEast1) [below of = insertSouth1, right of = insertSouth1, xshift = -0.75cm] {};
        \draw[--] (insertSouthWest1.center) to (insertSouth1.center) to (insertSouthEast1.center);
        
        \node(insertNorth2) [right of = define, xshift = -0.125cm, yshift = 0.25cm] {};
        \node(insertNorthWest2) [above of = insertNorth2, left of = insertNorth2, xshift = 0.75cm] {};
        \node(insertNorthEast2) [above of = insertNorth2, right of = insertNorth2, xshift = -0.75cm] {};
        \draw[--] (insertNorthWest2.center) to (insertNorth2.center) to (insertNorthEast2.center);
        
        \node(insertSouth2) [right of = define, xshift = -0.125cm, yshift = -0.25cm] {};
        \node(insertSouthWest2) [below of = insertSouth2, left of = insertSouth2, xshift = 0.75cm] {};
        \node(insertSouthEast2) [below of = insertSouth2, right of = insertSouth2, xshift = -0.75cm] {};
        \draw[--] (insertSouthWest2.center) to (insertSouth2.center) to (insertSouthEast2.center);


    \end{tikzpicture}
    }
    \caption{Best moments for user involvement (Adapted from \citeonline{moerland2021application}).}
    \label{fig:user_involvement}
\end{figure}

The authors described the application of the proposed approach to a use case. Three different designers initiated a cabin design. In the traditional method, the results were illustrated in a sketch, which could only present a glance of what the cabin would be, and in a 3D model, which had more details. However, any modification required a new rendering session and this could take hours, or even days. The same solution was also illustrated in a virtual reality environment. The sketch was inside the aircraft cabin, where the client or the stakeholder could draw and give their opinions from the beginning of the design process. The 3D models could be imported to increase the sketch's level of detail.

The use case showed some benefits and disadvantages of using virtual reality. The virtual reality helped to bring the client closer to the design team, allowing them to draw quick sketches in brainstorming gatherings. It was associated with a steep learning curve for the designers. Among the disadvantages, it was considered a high-cost tool, and its use for a long time was associated with nausea. 

The work of \citeonline{moerland2021application} is an example of how virtual reality can be used to bring the user into the design process. Similarly, in this work, virtual reality is explored to create a test environment where BVI users can try out the device under development, contributing to improving its usability. Differently from the work of \citeonline{moerland2021application}, in this work, users are not expected to feel sick, as it is usually associated with discrepancies between the motion of the image in the virtual environment and the motion perceived by the vestibular system of the user.