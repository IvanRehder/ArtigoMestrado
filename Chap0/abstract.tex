%% Abstract (Resumo em inglês)
\begin{abstract}
    Society has developed technology to create autonomous vehicles and to connect different devices and machinery to exchange data and optimize production efficiency.  With this technology, soon, it will be possible to achieve better methods to guide blind and visually impaired (BVI) users in their daily activities. We believe that the available products in the market have several limitations and do not satisfy BVI users and that one of the reasons behind this problem is that they are not members of the development team or are not consulted by these.

    The purpose of paper is to use virtual reality (VR) to test and evaluate different designs of BVI products. Also to verify if BVI and non-BVI users have the same mental demand and situation awareness when using assistive products. The idea is to use VR as a testing ground where a BVI user can try different assistive solutions in different scenarios. To illustrate the proposed method, a case study of navigation of BVI users inside a medical clinic is performed. 

    The scenes were made using Unity3D and the VR device was the Tobii Eye Tracking VR. Based on the current situation in the virtual environment, inputs are provided to the user using aural commands and haptics devices. To assess the mental workload, physiological sensors, from TEA Captiv T-Sens, are used. Among them, are an electrocardiogram sensor (ECG), to gather heart-rate and heart-rate variance data, and a galvanic skin response sensor (GSR), to collect skin conductance. Besides these sensors, the users are also expected to answer mental workload assessment tests and situation awareness questionnaires.

    Among the proposed method's expected benefits are the flexibility and agility to create different scenarios, and also the possibility to test all of them in the same physical room.
\end{abstract}

\begin{keyword}
    asda,asdfa,asdfasdf,asdfasdf
\end{keyword}