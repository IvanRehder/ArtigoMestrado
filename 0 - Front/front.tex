% Short title
\shorttitle{Virtual reality for the human-centred design of assistive devices}

% Short author
\shortauthors{E Villani et~al.}

% Main title of the paper
\title [mode = title]{Virtual reality for the human-centred design of assistive devices}

% Title footnote mark
% eg: \tnotemark[1]
%\tnotemark[1,2]

%% Title footnote 1.
%% eg: \tnotetext[1]{Title footnote text}
%% \tnotetext[<tnote number>]{<tnote text>} 
%%\tnotetext[1]{This document is the results of the research project funded by the National Science Foundation.}

%%\tnotetext[2]{The second title footnote which is a longer text matter to fill through the whole text width and overflow into another line in the footnotes area of the first page.}


% First author
%
% Options: Use if required
% eg: \author[1,3]{Author Name}[type=editor,
%       style=chinese,
%       auid=000,
%       bioid=1,
%       prefix=Sir,
%       orcid=0000-0000-0000-0000,
%       facebook=<facebook id>,
%       twitter=<twitter id>,
%       linkedin=<linkedin id>,
%       gplus=<gplus id>]
\author[1]{Emilia Villani}[type=editor,
                        prefix=Prof. Ph.D,
                        %degree = Ph.D,
                        %orcid=0000-0001-7511-2912
                        ]

% Corresponding author indication
\cormark[1]

% Footnote of the first author
%\fnmark[1]

% Email id of the first author
\ead{evillani@ita.br}

%  Credit authorship
%\credit{Conceptualization of this study, Methodology, Software}

% Second author
\author[1]{Edmar Thomaz da Silva}[prefix=Ph.D.,
                        %orcid=0000-0001-7511-2910,
                        %degree = Ph.D
                        ]

% Third author
\author[1]{Ivan de Souza Rehder}[prefix=M.Sc,
                        %orcid=0000-0001-7511-2910,
                        %degree = M.Sc
                        ]

%\fnmark[2]
%\cormark[2]
\ead{ivan@ita.org}

%\credit{Data curation, Writing - Original draft preparation}

% Corresponding author text
\cortext[cor1]{Corresponding author}
\cortext[cor2]{Principal corresponding author}

% Address/affiliation
\affiliation[1]{organization={Instituto Tecnológico de Aeronáutica},
    addressline={Praça Marechal Eduardo Gomes, 50},
    city={São José dos Campos},
    state={SP},
    postcode={12228-900}, 
    country={Brazil}}

% Footnote text
%\fntext[fn1]{This is the first author footnote. but is common to third
%  author as well.}
%\fntext[fn2]{Another author footnote, this is a very long footnote and
%  it should be a really long footnote. But this footnote is not yet
%  sufficiently long enough to make two lines of footnote text.}

% For a title note without a number/mark
%\nonumnote{This note has no numbers. In this work we demonstrate $a_b$
%  the formation Y\_1 of a new type of polariton on the interface
%  between a cuprous oxide slab and a polystyrene micro-sphere placed
%  on the slab.
%  }

% Here goes the abstract
\begin{abstract}
    Society has developed technology to create autonomous vehicles and to connect different devices and machinery to exchange data and optimize production efficiency.  With this technology, soon, it will be possible to achieve better methods to guide blind and visually impaired (BVI) users in their daily activities. We believe that the available products in the market have several limitations and do not satisfy BVI users and that one of the reasons behind this problem is that they are not members of the development team or are not consulted by these.

    The purpose of paper is to use virtual reality (VR) to test and evaluate different designs of BVI products. Also to verify if BVI and non-BVI users have the same mental demand and situation awareness when using assistive products. The idea is to use VR as a testing ground where a BVI user can try different assistive solutions in different scenarios. To illustrate the proposed method, a case study of navigation of BVI users inside a medical clinic is performed. 

    The scenes were made using Unity3D and the VR device was the Tobii Eye Tracking VR. Based on the current situation in the virtual environment, inputs are provided to the user using aural commands and haptics devices. To assess the mental workload, physiological sensors, from TEA Captiv T-Sens, are used. Among them, are an electrocardiogram sensor (ECG), to gather heart-rate and heart-rate variance data, and a galvanic skin response sensor (GSR), to collect skin conductance. Besides these sensors, the users are also expected to answer mental workload assessment tests and situation awareness questionnaires.

    Among the proposed method's expected benefits are the flexibility and agility to create different scenarios, and also the possibility to test all of them in the same physical room.
\end{abstract}

% Use if graphical abstract is present
% \begin{graphicalabstract}
% \includegraphics{figs/grabs.pdf}
% \end{graphicalabstract}

% Research highlights
%\begin{highlights}
%\item Research highlights item 1
%\item Research highlights item 2
%\item Research highlights item 3
%\end{highlights}

% Keywords
% Each keyword is seperated by \sep
\begin{keywords}
virtual reality \sep human factors \sep human machine interface
\end{keywords}