% ------------------------------------------------------------------
% Pacotes
% ------------------------------------------------------------------

% ABNT
\usepackage[onehalfspacing]{setspace}
\usepackage[brazil]{babel}
% Essencial
\usepackage{comment,enumerate}
% Matemática
\usepackage{amsmath,amsthm,amsfonts,amssymb,dsfont,mathtools}
\usepackage{cancel}      % needed to show canceled terms in equations
% ABNT
\usepackage{blindtext}

% ---
% Cores
% ---
\usepackage[dvipsnames]{xcolor}   % Code formating
    \definecolor{ccmBlue}{RGB}{20, 80, 200}
    \definecolor{ccmDBlue}{RGB}{25, 50, 120}
    \definecolor{ccmLBlue}{RGB}{170, 200, 230}
    \definecolor{ccmOrange}{RGB}{255, 100, 0}
    \definecolor{ccmRed}{RGB}{190, 0, 0}
    \definecolor{ccmLGray}{RGB}{210, 210, 210}
    \definecolor{ccmDGray}{RGB}{85, 85, 85}
    \definecolor{ccmWhite}{RGB}{250, 250, 250}
    
    % Ordem Paleta
    \definecolor{cor1}{RGB}{190, 000, 000}  %ccmRed
    \definecolor{cor2}{RGB}{025, 050, 120}  %ccmDBlue
    \definecolor{cor3}{RGB}{210, 210, 210}  %ccmLGray
    \definecolor{cor4}{RGB}{255, 100, 000}  %ccmOrange
    \definecolor{cor5}{RGB}{170, 200, 230}  %ccmLBlue
    \definecolor{cor6}{RGB}{085, 085, 085}  %ccmDGray
    \definecolor{cor7}{RGB}{020, 080, 200}  %ccmBlue

% ---
% Pacotes fundamentais 
% ---
\usepackage[utf8]{inputenc}                                                     % Codificacao do documento (conversão automática dos acentos)
\usepackage[T1]{fontenc}                                                        % Selecao de codigos de fonte.
\usepackage{enumitem}
\usepackage{lmodern}                                                            % Usa a fonte Latin Modern
\usepackage{indentfirst}		                                                % Indenta o primeiro parágrafo de cada seção.
\usepackage{nomencl} 			                                                % Lista de símbolos
\usepackage{color}				                                                % Controle das cores
\usepackage{graphicx}                                                           % Inclusão de gráficos
\usepackage{tabularx}                                                           % Inclusão de tábelas
%\usepackage{subfigure}
\usepackage{caption}
\usepackage{subcaption}
\usepackage{wrapfig}
\usepackage{float}
\usepackage{microtype} 			                                                % para melhorias de justificação
\usepackage{appendix}    % allows appendix section to be included
\usepackage{lscape}      % allows a page to be rendered in landscape mode
\usepackage{pdfpages}
\usepackage{listings}   % Code formating

% ---
% Criar hiperlinks
% ---
\usepackage[hidelinks,colorlinks=false]{hyperref}
\usepackage{url}         % formats URL addresses properly

% ---
% Tabelas
% ---
\usepackage{boldline}
\usepackage{multicol}    % allows text in multi columns
\usepackage{multirow}    % allows text in multi columns

% ---
% Tikz
% ---
\usepackage{pgf}
\usepackage{tikz}
    \usetikzlibrary{matrix,shapes.geometric,shapes.symbols,arrows.meta,positioning}
    \tikzset{>={Latex[round]}}

    % Varíaveis para plotar gráficos em barr
    %\newcommand{\tamX}{0}
    %\newcommand{\tamY}{0}
    %\newcommand{\distX}{0}
    %\newcommand{\distY}{0}
    %\newcommand{\largX}{0}
    %\newcommand{\largY}{0}
    %\newcommand{\altX}{0}
    %\newcommand{\altY}{0}
    %\newcommand{\msout}[1]{\text{\sout{\ensuremath{#1}}}}

% ---
% Criação de um indice
% ---
\usepackage{imakeidx}

% ---
% Outros
% ---
\usepackage{ae}
\usepackage{lipsum}                                                             % Lorem Ipsum
\usepackage{caption}                                                            % Customizar legendas
\usepackage{placeins}                                                           % FloatBarrier
\usepackage{lettrine}                                                           % Primeira letra do parágrafo grande
\usepackage{ulem}                                                               % Sublinhado

% ---
% Formatação dos capitulos e seções
% ---
\usepackage{titlesec}
    %\titleformat
    %    {\chapter} % command
    %    [display] % shape
    %    {\bfseries\Large\itshape} % format
    %    {} % label
    %    {0.5ex} % sep
    %    {
    %        \rule{\textwidth}{0.3pt}
    %        \centering
    %    } % before-code
    %    [
    %        \rule{\textwidth}{0.3pt}
    %    ] % after-code
    
    %\titleformat
    %    {\section} %command
    %    [] %shape
    %    {} %format
    %    {\thesection.} %label
    %    {} %sep
    %    {
    %        \vspace{1ex}
    %        \centering
    %    } %before code
    %    [
    %    \rule{\textwidth}{1pt}
    %    ] % after code
    
    %\titlespacing{\section}{12pc}{1.5ex plus .1ex minus .2ex}{1pc}

% ------------------------------------------------------------------
% Comandos
% ------------------------------------------------------------------
    
% Keywords command
%\providecommand{\keywords}[1]
%{
%  \small	
%  \textbf{\textit{Palavras-chave:}} #1
%}











