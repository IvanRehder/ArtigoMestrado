%\subsection{Motivation}

% Motivação: contextualização do problema ou da necessidade que motivou o trabalho. Essa seção deve preparar o leitor para a <Pergunta da Pesquisa>

According to the World Health Organisation (WHO) on the , there are at least 2.2 billion people with some visual impairment degree (\cite{world2019world}. Among them, 43,3 million are classified as blind and 295 million have moderate or severe vision impairment. In order to be fully integrated into our society, they rely on assistive devices, such as canes, braille speakers, among others \cite{bourne2021trends}. 

Although a range of products has already been proposed, incorporating different features, they do not entirely fulfill their aim. Among the problems, of the solutions available in the market, are the lack of practicality and portability, invasive and requiring too much effort to learn \cite{lozano2009electrotactile}.

The difficulty of using or learn how to use a device could be avoided if concepts from Human Factors, or Ergonomics, were analysed during the product’s development, using appropriate methods. The early application of these methods and tests could be a gamechanger for the success of the product's user experience \cite{wolf2019towards}.

Motivated by the dissatisfaction of blind people with the currently available products, this paper starts from the hypothesis that a human-factors-centred design of assistive devices for blind and visually impaired people (BVIs) requires the involvement of BVIs in the design process in order to evaluate the product under design. The user has to test the product under development to provide feedback for the design team to improve the product.

In order to approach this problem, this work proposes using virtual reality (VR) as a tool for creating virtual environments, where proof of concepts or prototypes of assistive devices could be tested by BVIs. VR can be used to create specific, immersive and interactive situations that could help the user to learn and train \cite{farrell2018learning}, and the the developers to create more user-friendly products.

In a virtual environment, as long as the BVI is wearing a locating system, s/he can navigate the environment. Any information about the scenario, such as the position of objects and their distances to the user, is known and could be extracted from the virtual platform. As a consequence the designer can test different ways of translating this information into inputs before actually implementing a prototype of the assistive device, providing a flexible, safe and easy way to have it evaluated by different users.

%\section{Virtual reality in the design process}
\label{sec:vr_cabin}
The use of virtual reality for design purposes is not new. The cabin design process is often said to be complex because it involves several stakeholders, each with his/her own set of preferences and requirements. \cite{moerland2021application} proposed to anticipate the involvement of the final users based on co-design. In their proposal, the users can influence the product's development from the beginning. However, for the involvement to happen, a communication channel needed to be established, and it was done using virtual reality. The use case showed some benefits and disadvantages of using virtual reality. The virtual reality helped to bring the client closer to the design team, allowing them to draw quick sketches in brainstorming gatherings. It was associated with a steep learning curve for the designers. Among the disadvantages, it was considered a high-cost tool, and its use for a long time was associated with nausea.

%\section{Virtual reality for BVI users}
\label{sec:vr_without_vision}
Motivated by the popularization of virtual reality technology, \cite{siu2020virtual} developed a white cane to be used by BVI users in a virtual environment. Their purpose was to make virtual reality applications available for BVI users. In order to evaluate their proposal, the authors performed an experiment where the participants had to play a “scavenger hunt” using an HTC Vive system. Among the relevant findings of \cite{siu2020virtual} is that not all the participants reacted the same to a particular stimulus. The vibration of the cane was considered confusing by some participants, while others were familiar with it. Another interesting observation was that, similar to what happens in the real world, it was easier for the participants to navigate in larger areas than in tight spaces. Moreover, the authors observed that the participants focused their attention on the primary task, without freely exploring the environment, which might have impacted the low time to achieve the goal and the low number of obstacle hits. 

%\section{Augmented reality for BVI users}
\label{sec:ar_without_vision}
\cite{kirner2011using} raised two questions, "How can blind people learn 3D concepts aiming to be able to convert explored 3D environments into pictures?" and "How can we develop a spatial audio tutor with augmented reality technology to make easy the understanding of 3D concepts by blind people?" and used not using virtual reality technology but augmented reality to answer them. They developed a augmented reality application to be a tutor for BVI users. The application used allowed BVI users to play audio streams that were associeated with spatial positions. The users learned 3D concepts and also were able to  perceive, understand and produce embossed pictures representing real and imaginary 3D scenes. Also they were able to understand descriptions of 3D scenes described by non-BVI people. The authors believe that this application can be evolved to explain other concepts such as colors, transparency, shades, etc.

%\section{Information for BVI navigation}
\label{sec:bradley_dunlop}
Bradley and Dunlop published two works (\citeyear{bradley2002investigating,bradley2005experimental}) about how BVI navigates and how much it is similar or different to how a sighted person navigates. The first work of Bradley and Dunlop was published in \citeyear{bradley2002investigating} and discussed which type of information BVI uses to navigate in an environment and how it compares to sighted people. The second they compared the perceived workload of BVI participants and sighted participants when they navigate using user-tailored information created with the results of the previous experiments \cite{bradley2005experimental}. The results showed that BVI users reached landmarks significantly quicker when given the information made for that group, but still longer than sighted users. Also it showed that BVI participants systematically have a higher workload than sighted participants and that BVI users did have a higher workload when guided by orientations provided by sighted people, as well as the sighted participants did with orientations from BVI.

%\subsection{Mental Workload (MWL)}
\label{sec:mental_workload}
\input{1 - Introducao/mental_workload}

%\subsection{Situation Awareness (SA)}
\label{sec:situation_awareness}
\input{1 - Introducao/situation_awareness}

%\section{Co-Design}
\label{sec:co_design}
\input{1 - Introducao/co_design}

This paper's main goal is the use of virtual reality as a tool for evaluating proofs of concept of assistive devices for blind and visually impaired people from a human-factors perspective. The purpose is to provide a flexible and easily configured way of testing different concepts of assistive devices in order to support an agile and user-centered development.

This goal is related to the following research questions, which are investigated in this work:

\begin{itemize}
   \item Is it possible to evaluate and compare concepts of assistive devices from a human factors perspective in a virtual environment? What are the main limitations of the use of a virtual reality environment? \label{itm:obj_first}
   \item Do non-BVI users, when deprived of their vision, similarly evaluate assistive devices as BVI users? \label{itm:obj_second}
\end{itemize}

% Delimitação da pesquisa: é o recorte do seu trabalho

The concepts of assistive devices presented as part of this work are used only as examples for investigating the research questions presented. The challenges related to their full development up to high Technology Readiness Levels (TRLs), as well as their feasibility as commercial products, are out of the scope of this work.

% Estrutura do texto

\subsection*{Structure of the text}

The next Section of this paper are organized as follows.

Section \ref{ch:metodologia} details the proposal of this paper describing how virtual reality could be used to integrate BVI users into the design process of assistive design. It illustrates the proposed method by applying it to evaluate three different assistive devices (audio guide, virtual cane and haptic belt), as well as their mixed-use, in the environment of a hospital reception. 

Section \ref{ch:resultados} describes the experiment designed to evaluate the paper's proposal and analyses the results in order to investigate the research questions of Section \ref{sec:objetivos}

Finally, section \ref{ch:conclusao} summarizes the main conclusions of this work and discusses future work.
