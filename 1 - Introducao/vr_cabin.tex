The use of virtual reality for design purposes is not new. The cabin design process is often said to be complex because it involves several stakeholders, each with his/her own set of preferences and requirements. \cite{moerland2021application} proposed to anticipate the involvement of the final users based on co-design. In their proposal, the users can influence the product's development from the beginning. However, for the involvement to happen, a communication channel needed to be established, and it was done using virtual reality. The use case showed some benefits and disadvantages of using virtual reality. The virtual reality helped to bring the client closer to the design team, allowing them to draw quick sketches in brainstorming gatherings. It was associated with a steep learning curve for the designers. Among the disadvantages, it was considered a high-cost tool, and its use for a long time was associated with nausea.