\subsubsection*{Final Remarks}

To summarize the conclusion obtained from the analysis of the data from blind participants, the audio method showed a lower score both for NASA-TLX mental demand and NASA-TLX global score. In contrast, the methods that include vibration achieved higher scores. This probably happened because the participants are already used to using sound to guide themselves, especially environmental sounds. The environment sounds used in the scenes were always the same (telephone ringing, laptop keyboard sounds, exterior noise, door opening and closing). The participants likely felt more relaxed when they only had to focus on the sounds around him/her. This is reinforced by the fact that, during the experiment with the audio method, half of the participants did not ask for any information, or the audio command option was used only a few times.

The fact that the haptic devices caused a higher workload is probably due to the fact that the users had to learn and get used to them. Besides, for being just conceptual, their precision was not as good as they were expecting. That explains why their results were not as good as the base or audio methods. The NASA-TLX results are correctly related to the satisfaction questionnaires, which scored them as the unsatisfied devices.

As expected, most of the variables from subjective questionnaires (NASA-TLX and SAGAT) show some influence of the rounds. On the other hand, the results from the physiological sensors did not show a clear tendency. 

The statistical analysis based on ANOVA tests confirmed some of the observations from the bar and box plots. However, in many cases, the residual distributions were not homogenous and the statistical analysis was affected by the small number of samples. 

All the blind participants showed great enthusiasm before, during and after the experiment. They also made several recommendations for both the virtual environment and the devices, such as:

\begin{itemize}
    \item The speakers of the HMD are not good enough to give them the precise location of the sound origin
    \item The HMD is too large and covers half of the participant's face. It gives them a strange sensation, since some of them use the air or the wind feeling on the face to give them hints about the location of walls or other high obstacles;
    \item The precision of the vibration for both the haptic belt and the virtual cane needs to be improved. It is not enough for them to use the devices. This problem is related to how the HMD sets the position of the user in the virtual environment. \\    
    \item The vibration from the haptic belt was not intense enough.
\end{itemize}


\subsubsection*{Final Remarks 2}

The comparison between the results from the blind participants and the sighted participants showed that there are significant differences in the evaluation performed by each group.

The sighted users evaluated the mental demand and other dimensions of NASA-TLX higher than blind ones. Also, blind participants were more familiar with audio methods and therefore gave a lower score to its mental demand. In the case of sighted participants, the method that received the lowest score was the virtual cane. 

The adapted SAGAT questionnaire showed a more significant influence of the round factor for blind participants, which significantly improved their situation awareness on the return round. In the case of sighted users, the difference between the rounds was not so striking. Also, the score achieved by sighted participants was lower than that of blind users, which was expected.

Another difference is that, for blind participants, it was possible to observe a difference between the methods that use vibration and those that do not. This difference was not clear for sighted participants. 

Besides these results, the sighted participants also gave feedback about the experiment. They felt considerably insecure when walking, even when hand-guided by another person. On the other hand, blind participants were already used to bumping their bodies when exploring new spaces. The sighted participants did not want that to happen and approached the furniture with caution. Similar to the blind participants, they also noticed the lack of precision of the haptic devices, but they did rely on them to navigate.