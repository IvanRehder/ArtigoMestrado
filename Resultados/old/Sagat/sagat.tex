\subsection{Adapted SAGAT}
\label{subsec:results_adapted_sagat}

In this subsection the Sagat questionnaire is analyzed. Its result's may give an ideia of the mental map the participant is drawing

The Shapiro–Wilk normality test on the Table \ref{tab:shapiro_sagat_score} shows that these data are normally distributed, with a p-value higher than 0.05, then it is possible to perform a t-test to guarantee that the "blind" sample is different then the "sight" sample and that is verified by the t-test's p-value that is lesser than 0.05 as show in the Table \ref{tab:ttest_sagat_score}.


\begin{table}[!htb]
\centering
\caption{Shapiro test p-value for the Sagat score for each method and visual impairment.}
\label{tab:shapiro_sagat_score}
\begin{tabular}{lr}
\toprule
                    Method &  Shapiro P-Value \\
\midrule
        Base blinded users &            0.102 \\
        Base sighted users &            0.074 \\
       Audio blinded users &            0.062 \\
       Audio sighted users &            0.777 \\
 Haptic Belt blinded users &            0.581 \\
 Haptic Belt sighted users &            0.909 \\
Virtual Cane blinded users &            0.587 \\
Virtual Cane sighted users &            0.852 \\
     Mixture blinded users &            0.755 \\
     Mixture sighted users &            0.096 \\
\bottomrule
\end{tabular}
\end{table}




\begin{table}[!htb]
\centering
\caption{T test p-value for the Sagat score on each method for blinded users versus sighted users.}
\label{tab:ttest_sagat_score}
\begin{tabular}{lr}
\toprule
      Method &  T-Test P-Value \\
\midrule
        Base &           0.467 \\
       Audio &           0.427 \\
 Haptic Belt &           0.278 \\
Virtual Cane &           0.731 \\
     Mixture &           0.004 \\
\bottomrule
\end{tabular}
\end{table}



The Table \ref{tab:sagat_average} presents these averages by each participant on each scenes.


\begin{table}[!htb]
\centering
\caption{Adapted Sagat average global score grouped by participant and guidance method.}
\label{tab:sagat_average}
\begin{tabular}{lrrrrrr}
\toprule
{} &   Base & Audio & \begin{tabular}[c]{@{}l@{}}Haptic\\ Belt\end{tabular} & \begin{tabular}[c]{@{}l@{}}Virtual\\ Cane\end{tabular} & Mixture & \begin{tabular}[c]{@{}l@{}}Visual\\ Condition\end{tabular} \\
Participant &        &       &                                                       &                                                        &         &                                                            \\
\midrule
001         &  10.00 &  5.25 &                                                  4.67 &                                                   3.83 &    5.50 &                                                      Sight \\
001C        &   6.25 &  6.00 &                                                  6.92 &                                                   5.67 &    4.50 &                                                      Blind \\
002C        &   6.00 &  4.75 &                                                  4.00 &                                                   5.50 &    7.38 &                                                      Blind \\
003         &  10.00 &  6.38 &                                                  6.62 &                                                   5.12 &    7.12 &                                                      Sight \\
003C        &   8.62 &  8.75 &                                                  8.00 &                                                   6.83 &    9.00 &                                                      Blind \\
004         &  10.00 &  7.50 &                                                  8.75 &                                                   7.12 &    7.62 &                                                      Sight \\
004C        &   8.25 &  6.00 &                                                  8.46 &                                                   6.12 &    7.75 &                                                      Blind \\
005         &  10.00 &  3.38 &                                                  3.08 &                                                   3.00 &    5.00 &                                                      Sight \\
\bottomrule
\end{tabular}
\end{table}



To be able to verify the impact of the methods on the "blind" sample, the Table \ref{tab:sagat_average_group} and the box plot on the Figure \ref{fig:boxplot_sagat_scene} presents the grouped average mental demands of the blinded and the sighted participants on each scenes and the box plot of the distribution of those averages.


\begin{table}[!htb]
\centering
\caption{Adapted Sagat average global score grouped by participant and visual Condition.}
\label{tab:sagat_average_group}
\begin{tabular}{lrrrrrr}
\toprule
{} &   Base & Audio & \begin{tabular}[c]{@{}l@{}}Haptic\\ Belt\end{tabular} & \begin{tabular}[c]{@{}l@{}}Virtual\\ Cane\end{tabular} &  Mixture \\
Visual Condition &        &       &                                                       &                                                        &          \\
\midrule
Blind            &   7.28 &  6.38 &                                                  6.84 &                                                   6.03 &    7.156 \\
Sight            &  10.00 &  5.62 &                                                  5.78 &                                                   4.77 &    6.312 \\
\bottomrule
\end{tabular}
\end{table}



\begin{figure}[!htb]
    \centering
    \resizebox{0.8\linewidth}{!}{
    \input{Resultados/Sagat/Figuras/boxplot_sagat_scene.pgf}    
    }
    \caption{Box plot average SAGAT score on each method by visual impairment.}
    \label{fig:boxplot_sagat_scene}
\end{figure}

The Figure \ref{fig:boxplot_sagat} and Table \ref{tab:sagat_average_group} is shown the global average of each group of participants

\begin{figure}[!htb]
    \centering
    \resizebox{0.8\linewidth}{!}{
    \input{Resultados/Sagat/Figuras/boxplot_sagat.pgf}    
    }
    \caption{Box plot of the average SAGAT score by the visual impairment.}
    \label{fig:boxplot_sagat}
\end{figure}

Through this figure is possible to see that ... 

Analysing these mental demand averages is possible to say that ...

\FloatBarrier