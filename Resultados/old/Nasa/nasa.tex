\subsection{NASA-TLX}
\label{subsec:results_nasa_tlx}

It is possible to analyze the mental workload using NASA-TLX two way. The first is to by analyzing only the mental demand scale and the second is by analyzing the NASA-TLX score, which is a average of the scales' rating.

\subsubsection{Analysis of the mental demand scale}

The Table \ref{tab:md_average} presents these averages by each participant on each scenes.


\begin{table}[!htb]
\centering
\caption{Mental demand average by participant and method.}
\label{tab:md_average}
\begin{tabular}{lrrrrrr}
\toprule
{} &  Base &  Audio & \begin{tabular}[c]{@{}l@{}}Haptic\\ Belt\end{tabular} & \begin{tabular}[c]{@{}l@{}}Virtual\\ Cane\end{tabular} & Mixture & \begin{tabular}[c]{@{}l@{}}Visual\\ Condition\end{tabular} \\
Participant &       &        &                                                       &                                                        &         &                                                            \\
\midrule
001         &  6.00 &  12.50 &                                                 12.00 &                                                   5.00 &    9.50 &                                                      Sight \\
001C        &  2.00 &   1.00 &                                                 12.00 &                                                   2.50 &    6.00 &                                                      Blind \\
002C        &  3.00 &   1.00 &                                                  1.00 &                                                  10.00 &    7.50 &                                                      Blind \\
003         &  1.50 &  15.00 &                                                 16.50 &                                                  13.50 &    9.00 &                                                      Sight \\
003C        &  4.00 &   3.00 &                                                  3.00 &                                                   5.00 &    1.00 &                                                      Blind \\
004         &  6.50 &  14.50 &                                                 17.50 &                                                  11.00 &   17.50 &                                                      Sight \\
004C        &  8.00 &  10.00 &                                                 14.50 &                                                   9.00 &   10.00 &                                                      Blind \\
005         &  2.00 &   5.00 &                                                 11.00 &                                                   8.00 &   12.50 &                                                      Sight \\
\bottomrule
\end{tabular}
\end{table}



%% Blind users vs Sight User

The Shapiro–Wilk normality test on the Table \ref{tab:shapiro_mental_demand} shows that these data are normally distributed, with a p-value higher than 0.05, then it is possible to perform a t-test to guarantee that the "blind" sample is different then the "sight" sample and that is verified by the t-test's p-value that is lesser than 0.05 as show in the Table \ref{tab:ttest_mental_demand}.


\begin{table}[!htb]
\centering
\caption{Shapiro test p-value for the mental demand for each method and visual impairment.}
\label{tab:shapiro_mental_demand}
\begin{tabular}{lr}
\toprule
                    Method &  Shapiro P-Value \\
\midrule
        Base blinded users &            0.286 \\
        Base sighted users &            0.155 \\
       Audio blinded users &            0.041 \\
       Audio sighted users &            0.443 \\
 Haptic Belt blinded users &            0.498 \\
 Haptic Belt sighted users &            0.932 \\
Virtual Cane blinded users &            0.124 \\
Virtual Cane sighted users &            0.109 \\
     Mixture blinded users &            0.027 \\
     Mixture sighted users &            0.183 \\
\bottomrule
\end{tabular}
\end{table}




\begin{table}[!htb]
\centering
\caption{T test p-value for the mental demand on each method for blinded users versus sighted users.}
\label{tab:ttest_mental_demand}
\begin{tabular}{lr}
\toprule
      Method &  T-Test P-Value \\
\midrule
        Base &           0.244 \\
       Audio &           0.482 \\
 Haptic Belt &           0.835 \\
Virtual Cane &           0.113 \\
     Mixture &           0.425 \\
\bottomrule
\end{tabular}
\end{table}




This means that the mental demand of...

%% Methods

To be able to verify the impact of the methods on the "blind" sample, the Table \ref{tab:md_average_group} and the box plot on the Figure \ref{fig:boxplot_md_scene} presents the grouped average mental demands of the blinded and the sighted participants on each scenes and the box plot of the distribution of those averages.


\begin{table}[!htb]
\centering
\caption{Mental demand average grouped by participant and visual condition}
\label{tab:md_average_group}
\begin{tabular}{lrrrrrr}
\toprule
{} &  Base &  Audio & \begin{tabular}[c]{@{}l@{}}Haptic\\ Belt\end{tabular} & \begin{tabular}[c]{@{}l@{}}Virtual\\ Cane\end{tabular} &  Mixture \\
Visual Condition &       &        &                                                       &                                                        &          \\
\midrule
Blind            &  4.25 &   3.75 &                                                  7.62 &                                                   6.62 &    6.125 \\
Sight            &  4.00 &  11.75 &                                                 14.25 &                                                   9.38 &   12.125 \\
\bottomrule
\end{tabular}
\end{table}




\begin{figure}[!htb]
    \centering
    \resizebox{0.8\linewidth}{!}{
    \input{Resultados/Nasa/Figuras/boxplot_md_scene.pgf}    
    }
    \caption{Box plot average mental demand on each method by visual impairment.}
    \label{fig:boxplot_md_scene}
\end{figure}


The Figure \ref{fig:boxplot_md} and Table \ref{tab:md_average_group} is shown the global average of each group of participants

\begin{figure}[!htb]
    \centering
    \resizebox{0.8\linewidth}{!}{
    \input{Resultados/Nasa/Figuras/boxplot_md.pgf}    
    }
    \caption{Box plot of the average mental demand by the visual impairment.}
    \label{fig:boxplot_md}
\end{figure}

Through this figure is possible to see that ... 

Analysing these mental demand averages is possible to say that ...

\FloatBarrier

\subsubsection{Analysis of the NASA-TLX score}

The Shapiro–Wilk normality test in the Table \ref{tab:shapiro_mental_demand} shows that these data are normally distributed, with a p-value higher than 0.05, then it is possible to perform a t-test to guarantee that the "blind" sample is different then the "sight" sample and that is verified by the t-test's p-value that is lesser than 0.05.


\begin{table}[!htb]
\centering
\caption{Shapiro test p-value for the NASA score for each method and visual impairment.}
\label{tab:shapiro_nasa_score}
\begin{tabular}{lr}
\toprule
                    Method &  Shapiro P-Value \\
\midrule
        Base blinded users &            0.969 \\
        Base sighted users &            0.489 \\
       Audio blinded users &            0.723 \\
       Audio sighted users &            0.973 \\
 Haptic Belt blinded users &            0.681 \\
 Haptic Belt sighted users &            0.049 \\
Virtual Cane blinded users &            0.778 \\
Virtual Cane sighted users &            0.215 \\
     Mixture blinded users &            0.493 \\
     Mixture sighted users &            0.139 \\
\bottomrule
\end{tabular}
\end{table}




\begin{table}[!htb]
\centering
\caption{T test p-value for the NASA score on each method for blinded users versus sighted users.}
\label{tab:ttest_nasa_score}
\begin{tabular}{lr}
\toprule
      Method &  T-Test P-Value \\
\midrule
        Base &           0.577 \\
       Audio &           0.022 \\
 Haptic Belt &           0.305 \\
Virtual Cane &           0.286 \\
     Mixture &           0.809 \\
\bottomrule
\end{tabular}
\end{table}



The Table \ref{tab:nasa_average} presents these averages by each participant on each scenes and the Figure \ref{fig:boxplot_nasa_scene} shows these data plotted.


\begin{table}[!htb]
\centering
\caption{NASA-TLX score grouped by participant and method.}
\label{tab:nasa_average}
\begin{tabular}{lrrrrrr}
\toprule
{} &   Base &   Audio & \begin{tabular}[c]{@{}l@{}}Haptic\\ Belt\end{tabular} & \begin{tabular}[c]{@{}l@{}}Virtual\\ Cane\end{tabular} & Mixture & \begin{tabular}[c]{@{}l@{}}Visual\\ Condition\end{tabular} \\
Participant &        &         &                                                       &                                                        &         &                                                            \\
\midrule
001         &  7.9\% &  10.6\% &                                                10.3\% &                                                  6.6\% &   9.2\% &                                                      Sight \\
001C        &  4.5\% &   4.0\% &                                                 7.8\% &                                                  4.8\% &   6.2\% &                                                      Blind \\
002C        &  5.4\% &   4.8\% &                                                 4.8\% &                                                  8.0\% &   6.1\% &                                                      Blind \\
003         &  4.6\% &   8.2\% &                                                 9.9\% &                                                  8.7\% &   5.7\% &                                                      Sight \\
003C        &  4.0\% &   3.9\% &                                                 4.5\% &                                                  5.1\% &   3.5\% &                                                      Blind \\
004         &  6.8\% &  13.3\% &                                                12.8\% &                                                 11.2\% &  14.0\% &                                                      Sight \\
004C        &  9.2\% &   9.6\% &                                                12.2\% &                                                  9.5\% &  10.9\% &                                                      Blind \\
005         &  5.0\% &   7.7\% &                                                 8.8\% &                                                  7.8\% &   7.8\% &                                                      Sight \\
\bottomrule
\end{tabular}
\end{table}



%\begin{table}[!htb]
%\centering
%\caption{NASA-TLX score grouped by participant and visual impairment.}
%\label{tab:nasa_average}
%\begin{tabular}{lrrrrr}
%\toprule
%{} &  Base &  Audio &  Haptic Belt &  Virtual Cane &  Mixture \\
%Visual Impairment &       &        &              &               &          \\
%\midrule
%0                 & 2.000 & 12.000 &       12.000 &         2.500 &    6.000 \\
%1                 & 6.000 & 12.500 &       12.000 &         5.000 &    9.500 \\
%\bottomrule
%\end{tabular}
%\end{table}

To be able to verify the impact of the methods on the "blind" sample, the Table \ref{tab:nasa_average_group} and the box plot on the Figure \ref{fig:boxplot_nasa_scene} presents the grouped average NASA-TLX scores of the blinded and the sighted participants on each scenes and the box plot of the distribution of those averages.


\begin{table}[!htb]
\centering
\caption{Average NASA-TLX score grouped by participant and visual Condition}
\label{tab:nasa_average_group}
\begin{tabular}{lrrrrrr}
\toprule
{} &  Base & Audio & \begin{tabular}[c]{@{}l@{}}Haptic\\ Belt\end{tabular} & \begin{tabular}[c]{@{}l@{}}Virtual\\ Cane\end{tabular} &  Mixture \\
Visual Condition &       &       &                                                       &                                                        &          \\
\midrule
Blind            &  5.79 &  5.58 &                                                  7.31 &                                                   6.85 &    6.688 \\
Sight            &  6.06 &  9.96 &                                                 10.46 &                                                   8.56 &    9.167 \\
\bottomrule
\end{tabular}
\end{table}



%\begin{table}[!htb]
%\centering
%\caption{Average NASA-TLX score by the blinded and sighted participants on each method.}
%\label{tab:nasa_average_group}
%\begin{tabular}{lrrrrrr}
%{}
%\end{tabular}
%\end{table}

\begin{figure}[!htb]
    \centering
    \resizebox{0.8\linewidth}{!}{
    \input{Resultados/Nasa/Figuras/boxplot_nasa_scene.pgf}    
    }
    \caption{Box plot average NASA-TLX score on each method by visual impairment.}
    \label{fig:boxplot_nasa_scene}
\end{figure}

\begin{figure}[!htb]
    \centering
    \resizebox{0.8\linewidth}{!}{
    \input{Resultados/Nasa/Figuras/boxplot_nasa.pgf}    
    }
    \caption{Box plot of the average NASA-TLX score by the visual impairment.}
    \label{fig:boxplot_nasa}
\end{figure}

The Figure \ref{fig:boxplot_nasa} and Table \ref{tab:nasa_average_group} is shown the global average of each group of participants

Through this figure is possible to see that ... 

Analysing these NASA-TLX score averages is possible to say that ...

\FloatBarrier