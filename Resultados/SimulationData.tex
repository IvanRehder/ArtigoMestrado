%Unity3D was programmed to record two different information as performance evaluation, they were:
%
%\begin{itemize}
%    \item \nameref{subsec:results_time};
%    \item \nameref{subsec:results_collsions}.
%\end{itemize}

Unity3D was programmed to record the time that each user spent in each scene. It is expected that the time analysis will show the following observation:

\begin{itemize}
    \item The scene made with the white cane would be the fastest and with the less number of impacts; \\ 
    Since the participant is already used to this method, it is safe to assume that with the others methods the participant would go slower and hit more furniture on the way.
    \item Comparing both scenes made with the same method, the second one would have the fastest and with less impact; \\
    Not only this is expected but also is the intention on having two scenes with each method.
\end{itemize}

\subsection{Time elapsed on each scene}
\label{subsec:results_collsions}


The data collected from the participants are shown in the Table \ref{tab:duracao_average}.


\begin{table}[!htb]
\centering
\caption{Duration grouped by participant and guidance method (in minutes).}
\label{tab:duracao_average}
\begin{tabular}{lllrrrrr}
\toprule
    &       &        &   Base &  Audio & \begin{tabular}[c]{@{}l@{}}Haptic\\ Belt\end{tabular} & \begin{tabular}[c]{@{}l@{}}Virtual\\ Cane\end{tabular} & Mixture \\
Participant & \begin{tabular}[c]{@{}l@{}}Visual\\ Condition\end{tabular} & Round &        &        &                                                       &                                                        &         \\
\midrule
001 & Sight & First &  10:18 &  13:05 &                                                  6:42 &                                                   6:52 &    7:54 \\
    &       & Return &  12:38 &   6:25 &                                                  7:41 &                                                  10:28 &    5:21 \\
001C & Blind & First &   2:11 &   6:00 &                                                 10:41 &                                                   9:02 &    7:42 \\
    &       & Return &  11:21 &   7:41 &                                                  6:06 &                                                   5:36 &    6:10 \\
002C & Blind & First &   2:02 &   6:17 &                                                  4:32 &                                                   7:34 &    4:08 \\
    &       & Return &  13:32 &   8:06 &                                                  8:02 &                                                   3:35 &    3:57 \\
003 & Sight & First &   8:06 &   2:14 &                                                  2:51 &                                                   4:21 &    8:11 \\
    &       & Return &   4:11 &  15:25 &                                                  6:50 &                                                   5:25 &    4:18 \\
003C & Blind & First &   2:40 &  11:16 &                                                  8:04 &                                                   5:20 &    5:42 \\
    &       & Return &   6:38 &   4:59 &                                                  4:00 &                                                   8:52 &    5:32 \\
004 & Sight & First &   2:30 &   5:59 &                                                  4:16 &                                                   1:44 &    6:22 \\
    &       & Return &   6:39 &   4:32 &                                                  5:11 &                                                   5:25 &    6:16 \\
004C & Blind & First &   2:30 &   6:26 &                                                  4:23 &                                                   5:04 &    3:54 \\
    &       & Return &   8:29 &   6:48 &                                                 11:25 &                                                   4:29 &    6:24 \\
005 & Sight & First &   2:33 &   6:58 &                                                  5:34 &                                                   5:09 &    7:52 \\
    &       & Return &   8:16 &   8:46 &                                                  4:25 &                                                   6:45 &    3:00 \\
\bottomrule
\end{tabular}
\end{table}



The Table \ref{tab:duracao_min_average_group} show the average time in each method grouped by visual condition and they are plotted in the Figure \ref{fig:barplot_duration_scene_blind} and \ref{fig:barplot_duration_scene_sight}. The Figure \ref{fig:barplot_duration_scene_blind} shows that there is no pattern in time variation in the rounds. The Figure \ref{fig:barplot_duration_scene_sight} shows that there is an increase in the time in the "Return" round by every method, with the exception of the "Mixture" method. 

%
\begin{table}[!htb]
\centering
\caption{Average duration grouped by participant and guidance method (in minutes).}
\label{tab:duracao_average_scene}
\begin{tabular}{lrrrrrl}
\toprule
{} &   Base & Audio & Haptic Belt & Virtual Cane & Mixture & Visual Condition \\
Participant &        &       &             &              &         &                  \\
\midrule
001         &  11:28 &  9:45 &        7:12 &         8:40 &    6:38 &            Sight \\
001C        &   6:46 &  6:50 &        8:23 &         7:19 &    6:56 &            Blind \\
002C        &   7:47 &  7:11 &        6:17 &         5:34 &    4:02 &            Blind \\
003         &   6:09 &  8:50 &        4:51 &         4:53 &    6:14 &            Sight \\
003C        &   4:39 &  8:08 &        6:02 &         7:06 &    5:37 &            Blind \\
004         &   4:35 &  5:16 &        4:44 &         3:35 &    6:19 &            Sight \\
004C        &   5:30 &  6:37 &        7:54 &         4:46 &    5:09 &            Blind \\
005         &   5:25 &  7:52 &        4:59 &         5:57 &    5:26 &            Sight \\
\bottomrule
\end{tabular}
\end{table}



\begin{table}[!htb]
\centering
\caption{Duration difference grouped by participant and visual condition}
\label{tab:duracao_min_average_group}
\begin{tabular}{lrrrrr}
\toprule
{} &  Base & Audio & Haptic Belt & Virtual Cane & Mixture \\
Visual Condition &       &       &             &              &         \\
\midrule
Blind            &  6:10 &  7:11 &        7:09 &         6:11 &    5:26 \\
Sight            &  6:54 &  7:56 &        5:26 &         5:46 &    6:09 \\
\bottomrule
\end{tabular}
\end{table}




\begin{figure}[!htb]
    \centering
    \begin{minipage}{\textwidth}
        \centering
        \includegraphics[width = 0.8\linewidth]{Resultados/Tempo/Figuras/png/barplot_duration_scene_blind.png}
        \caption{Bar plot of the average time of the blind participants on each method.}
        \label{fig:barplot_duration_scene_blind}
    \end{minipage}
    \begin{minipage}{\textwidth}
        \centering
        \includegraphics[width = 0.8\linewidth]{Resultados/Tempo/Figuras/png/barplot_duration_scene_sight.png}
        \caption{Bar plot of the average time of sighted participants on each method.}
        \label{fig:barplot_duration_scene_sight}
    \end{minipage}
\end{figure}


In the Figure \ref{fig:boxplot_duration_scene} is plotted the time of each participant and it shows that there could be some difference in the time between the methods, but that would only be assured with a hypothesis test. 

%
\begin{table}[!htb]
\centering
\caption{Duration difference grouped by participant and guidance method.}
\label{tab:duracao_var}
\begin{tabular}{lrrrrrr}
\toprule
{} &     Base &    Audio & \begin{tabular}[c]{@{}l@{}}Haptic\\ Belt\end{tabular} & \begin{tabular}[c]{@{}l@{}}Virtual\\ Cane\end{tabular} &  Mixture & \begin{tabular}[c]{@{}l@{}}Visual\\ Condition\end{tabular} \\
Participant &          &          &                                                       &                                                        &          &                                                            \\
\midrule
001         &   22.6\% &  -50.9\% &                                                14.7\% &                                                 52.4\% &  -32.2\% &                                                      Sight \\
001C        &  419.6\% &   28.2\% &                                               -42.9\% &                                                -37.9\% &  -20.0\% &                                                      Blind \\
002C        &  563.4\% &   28.9\% &                                                77.3\% &                                                -52.6\% &   -4.2\% &                                                      Blind \\
003         &  -48.3\% &  587.1\% &                                               139.6\% &                                                 24.5\% &  -47.5\% &                                                      Sight \\
003C        &  148.9\% &  -55.8\% &                                               -50.3\% &                                                 66.3\% &   -2.8\% &                                                      Blind \\
004         &  165.0\% &  -24.1\% &                                                21.5\% &                                                209.6\% &   -1.4\% &                                                      Sight \\
004C        &  237.4\% &    5.6\% &                                               160.5\% &                                                -11.5\% &   63.7\% &                                                      Blind \\
005         &  222.8\% &   25.9\% &                                               -20.6\% &                                                 30.9\% &  -61.9\% &                                                      Sight \\
\bottomrule
\end{tabular}
\end{table}



The Table \ref{tab:duracao_var_group} show the time variation in each method grouped by visual condition and Figure \ref{fig:barplot_duration_global} these data is plotted. The table shows a noticeable difference between the two groups. The Figure \ref{fig:barplot_duration_global} shows that the global average of the groups in all scenes were almost the same.


\begin{table}[!htb]
\centering
\caption{Duration difference grouped by participant and visual Condition.}
\label{tab:duracao_var_group}
\begin{tabular}{lrrrrr}
\toprule
{} &    Base &   Audio & Haptic Belt & Virtual Cane & Mixture \\
Visual Condition &         &         &             &              &         \\
\midrule
Blind            &  342.3\% &    1.7\% &       36.1\% &        -8.9\% &    9.2\% \\
Sight            &   90.5\% &  134.5\% &       38.8\% &        79.4\% &  -35.7\% \\
\bottomrule
\end{tabular}
\end{table}



\begin{figure}[!htb]
    %\centering
    \begin{minipage}{.45\linewidth}
        \centering
        \includegraphics[width = \linewidth]{Resultados/Tempo/Figuras/png/boxplot_time_scene.png}
        %\resizebox{\linewidth}{!}{
        %\input{Resultados/Tempo/Figuras/boxplot_duration_scene.pgf}
        %}
        \caption{Boxplot of the average time of each group on each method.}
        \label{fig:boxplot_duration_scene}
    \end{minipage}
    \begin{minipage}{.1\linewidth}
        \hfill
    \end{minipage}
    \begin{minipage}{.45\linewidth}
        \centering
        \vspace{1.8cm}
        \includegraphics[width = \linewidth]{Resultados/Tempo/Figuras/png/barplot_duration_global.png}
        %\resizebox{\linewidth}{!}{
        %\input{Resultados/Tempo/Figuras/barplot_duration_global.pgf}
        %}
        \caption{Barplot of the average time of each group.}
        \label{fig:barplot_duration_global}
    \end{minipage}
\end{figure}

For more correct analysis, one should use statistical methods to analyze. So hypothesis tests were used, but the first step in this analysis is to check if the sample has a normal distribution. 

The Table \ref{tab:shapiro_duration} shows the Shapiro Wilk test p-value. If this value is higher than 0.05, then the sample is normally distributed. The table \ref{tab:shapiro_duration} indicates that the p-values of the time averages are normally distributed hence the steps that follow are allowed to be used.

%
\begin{table}[!htb]
\centering
\caption{Shapiro test p-value for the duration of participant in each method.}
\label{tab:shapiro_duration}
\begin{tabular}{lr}
\toprule
                    Method &  Shapiro P-Value \\
\midrule
       Audio blinded users &            0.175 \\
       Audio blinded users &            0.552 \\
 Haptic Belt blinded users &            0.276 \\
 Haptic Belt blinded users &            0.915 \\
Virtual Cane blinded users &            0.315 \\
Virtual Cane blinded users &            0.580 \\
     Mixture blinded users &            0.377 \\
     Mixture blinded users &            0.437 \\
\bottomrule
\end{tabular}
\end{table}



The Table \ref{tab:ttest_duration} shows the T-test p-value between the time average of the blind sample and the time average of the sight sample. If this value is higher than 0.05, it means that there is no statistical differences between the samples and that both samples had the same time performance. The table \ref{tab:ttest_duration} indicates the time of both the blind and the sighted users are statistically the same.

%
\begin{table}[!htb]
\centering
\caption{T test p-value for the duration for blinded users versus sighted users.}
\label{tab:ttest_duration}
\begin{tabular}{lr}
\toprule
      Method &  T-Test P-Value \\
\midrule
        Base &           0.683 \\
       Audio &           0.498 \\
 Haptic Belt &           0.085 \\
Virtual Cane &           0.746 \\
     Mixture &           0.313 \\
\bottomrule
\end{tabular}
\end{table}



\begin{table}[!htb]
    \begin{minipage}{.45\linewidth}
        
\centering
\begin{tabular}{lr}
\toprule
                    Method &  Shapiro P-Value \\
\midrule
       Audio blinded users &            0.175 \\
       Audio blinded users &            0.552 \\
 Haptic Belt blinded users &            0.276 \\
 Haptic Belt blinded users &            0.915 \\
Virtual Cane blinded users &            0.315 \\
Virtual Cane blinded users &            0.580 \\
     Mixture blinded users &            0.377 \\
     Mixture blinded users &            0.437 \\
\bottomrule
\end{tabular}

    \end{minipage}
    \hfill
    \begin{minipage}{.45\linewidth}
        \vspace{-1.5cm}
        
\centering
\begin{tabular}{lr}
\toprule
      Method &  T-Test P-Value \\
\midrule
        Base &           0.683 \\
       Audio &           0.498 \\
 Haptic Belt &           0.085 \\
Virtual Cane &           0.746 \\
     Mixture &           0.313 \\
\bottomrule
\end{tabular}

    \end{minipage}
\end{table}

The Table \ref{tab:blockedanova_duration_var} shows the Anova test p-value of the blind time averages between the guidance methods. If this value is higher than 0.05, there is at least one method that has no statistical difference between one from the other methods. The table \ref{tab:blockedanova_duration_var} indicates that at all of the time observations are different.

%
\begin{table}[!htb]
\centering
\caption{Duration difference grouped by participant and guidance method.}
\label{tab:duracao_var}
\begin{tabular}{lrrrrrr}
\toprule
{} &     Base &    Audio & \begin{tabular}[c]{@{}l@{}}Haptic\\ Belt\end{tabular} & \begin{tabular}[c]{@{}l@{}}Virtual\\ Cane\end{tabular} &  Mixture & \begin{tabular}[c]{@{}l@{}}Visual\\ Condition\end{tabular} \\
Participant &          &          &                                                       &                                                        &          &                                                            \\
\midrule
001         &   22.6\% &  -50.9\% &                                                14.7\% &                                                 52.4\% &  -32.2\% &                                                      Sight \\
001C        &  419.6\% &   28.2\% &                                               -42.9\% &                                                -37.9\% &  -20.0\% &                                                      Blind \\
002C        &  563.4\% &   28.9\% &                                                77.3\% &                                                -52.6\% &   -4.2\% &                                                      Blind \\
003         &  -48.3\% &  587.1\% &                                               139.6\% &                                                 24.5\% &  -47.5\% &                                                      Sight \\
003C        &  148.9\% &  -55.8\% &                                               -50.3\% &                                                 66.3\% &   -2.8\% &                                                      Blind \\
004         &  165.0\% &  -24.1\% &                                                21.5\% &                                                209.6\% &   -1.4\% &                                                      Sight \\
004C        &  237.4\% &    5.6\% &                                               160.5\% &                                                -11.5\% &   63.7\% &                                                      Blind \\
005         &  222.8\% &   25.9\% &                                               -20.6\% &                                                 30.9\% &  -61.9\% &                                                      Sight \\
\bottomrule
\end{tabular}
\end{table}



%\input{Resultados/Tempo/Tabelas/anova_duration_var.tex}

%
\begin{table}[!htb]
\centering
\caption{Anova p-value for the duration difference of each method for blinded users.}
\label{tab:blockedanova_duration_var}
\begin{tabular}{lrrrrr}
\toprule
Source & P-Value \\
\midrule
Method & 0.002** \\
\bottomrule
\end{tabular}
\end{table}




\begin{table}[!htb]
\centering
\caption{Anova p-value for the duration difference of each method for blinded users.}
\label{tab:blockedanova_duration_var}
\begin{tabular}{lrrrrr}
\toprule
Source & P-Value \\
\midrule
Method & 0.002** \\
\bottomrule
\end{tabular}
\end{table}



%The Table \ref{tab:lsd_duration} presents a pairwise Fisher LSD test of the blind time average between all the guidance methods. If the resulted p-value is higher than 0.05 then the pair is statistically the same. The Table \ref{tab:lsd_duration} shows the conclusion of each p-value, and that is that all the methods provoke a different time reaction than the time on the "Base" method, but there is no difference between them. only the "Virtual Cane" would be considered similar to the "Base" method, and looking at the Figure \ref{fig:boxplot_duration_scene} above one can see that both distributions are very similar. As for the other methods, the one that improved the time spent at each scene would be the "Mixture". 
%
%\input{Resultados/Tempo/Tabelas/lsd_duration.tex}

Considering the on Table \ref{tab:ttest_duration}, the duration of the "sight" sample is similar to the "blind" sample and considering the conclusion from the ANOVA test and the Tables \ref{tab:duracao_min_average_group} and \ref{tab:duracao_var_group}, the methods that had the best time efficiency was the "Mixture" and the "Virtual Cane", and the least one was the "Audio" method.

Despite all these results above, it is noticeable some outliers in the data, especially in the first participants, when the most minor procedure errors, such as the one to stop the simulation, hence stopping the timer, had happened.


\FloatBarrier


%\subsection{Number of collisions of each participant}
%\label{subsec:results_time}
%
%As it has done with the time results, a Shapiro–Wilk normality test with the number of collisions %has a p-value higher than 0.05 showing that these data are normally distributed. The t-test of %"blind" sample versus the "sight" sample has a p-values lesser 0.05, than these sample are %statistically different.
%
%The Table \ref{tab:collisions_average} presents the average collisions of each participant that %happened on each scenes and the Figure \ref{fig:collisions_average} shows these data plotted.
%
%\begin{table}[!htb]
%\centering
%\caption{Average collisions on each method by each participant.}
%\label{tab:collisions_average}
%\begin{tabular}{lrrrrrr}
%{}
%\end{tabular}
%\end{table}
%
%\begin{figure}[!htb]
%    \centering
%    \includegraphics{}
%    \caption{Plotted average collisions on each method by each participant.}
%    \label{fig:collisions_average}
%\end{figure}
%
%To be able to verify the impact of the methods on the "blind" sample, the Table %\ref{tab:collisions_average_group} and the box plot on the Figure \ref{fig:collisions_blind_boxplot} %and \ref{fig:collisions_sight_boxplot} presents the grouped averages of the blinded and the sighted %participants on each scenes and the box plot of the distribution of those averages.
%
%\begin{table}[!htb]
%\centering
%\caption{Average collisions by the blinded and sighted participants on each method.}
%\label{tab:collisions_average_group}
%\begin{tabular}{lrrrrrr}
%{}
%\end{tabular}
%\end{table}
%
%\begin{figure}[!htb]
%    \centering
%    \includegraphics{}
%    \caption{Box plot of the average collisions on each method by the blind participants.}
%    \label{fig:collisions_blind_boxplot}
%\end{figure}
%
%Through this figure is possible to see that ... 
%
%\begin{figure}[!htb]
%    \centering
%    \includegraphics{}
%    \caption{Box plot of the average collisions on each method by the sight participants.}
%    \label{fig:collisions_sight_boxplot}
%\end{figure}
%
%Analysing these time averages is possible to say that ...



\FloatBarrier